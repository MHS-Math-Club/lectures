\documentclass{beamer}
\usetheme{default}

\title{RSA Encryption: Math Club Notes}
\author{Markus Hoehn}
\date{}

\begin{document}

\begin{frame}
    \titlepage
\end{frame}


\begin{frame}{Overview}
    \begin{itemize}
    \item RSA (Rivest–Shamir–Adleman) is one of the oldest and most widely used public-key cryptosystems.
    \item Public-key cryptography involves individuals generating a private key (kept to yourself) and a public key, which they share with others.
    \item Messages can be encrypted using someone's public key, and only the recipient possessing the corresponding private key can decrypt them.
    \end{itemize}
\end{frame}

\begin{frame}{Modular Arithmetic}
    \begin{itemize}
        \item Modular arithmetic is a system for working with integers under a specified modulus.
        \item The expression $x \equiv a \pmod{N}$ means that $x$ is congruent to $a$ modulo $N$. In other words, $x$ and $a$ yield the same remainder when divided by $N$.
        \item For example, $15 \equiv -9 \pmod{3}$ since both yield a remainder of $0$ when divided by $3$.
        \item Clock arithmetic is a specific case of modular arithmetic with a modulus of $12$. For instance, if it is $9:00$ now, then in $6$ hours, the time will be $15:00$, which is congruent to $3:00$ modulo $12$.
    \end{itemize}
\end{frame}

\begin{frame}{Euler's Totient Function}
    \begin{itemize}
        \item Euler's totient function, denoted $\phi(n)$, counts the number of positive integers $m$ such that $1 \leq m < n$ and $\text{gcd}(m, n) = 1$.
        \item Two numbers are said to be coprime if their greatest common divisor (gcd) is $1$, indicating they share no common factors other than $1$. Notably, primes are coprime to all preceding positive integers, so $\phi(p) = p - 1$ for prime $p$.
        \item For example, $\phi(15) = 8$ since $\{1, 2, 4, 7, 8, 11, 13, 14\}$ are coprime to $15$, leaving out $\{3, 5, 6, 9, 10, 12\}$.
        \item It is a known fact that $\phi(nm) = \phi(n)\phi(m)$ when $n$ and $m$ are coprime.
    \end{itemize}
\end{frame}

\begin{frame}{Euler's Theorem}
    Let $n$ be a positive integer, and let $a$ be an integer coprime to $n$. Then, according to Euler's theorem:
    \[
        a^{\phi(n)} \equiv 1 \pmod{n}    
    \]
    For example, for any number $a$ coprime to $15$, since $\phi(15) = 8$, we have $a^{8} \equiv 1 \pmod{10}$. 
\end{frame}

\begin{frame}{Modular Inverses}
    \begin{itemize}
        \item The modular inverse of $a$ modulo $m$ is the integer $x$ such that $ax \equiv 1 \pmod{m}$. If $a$ and $m$ are coprime, then there exists an integer $x$ that serves as the modular inverse of $a$.
        \item We can use the Extended Euclidean algorithm to compute modular inverses, which is an algorithm relying on the computationally easy multiplication.
    \end{itemize}
\end{frame}

\begin{frame}{Factoring Problem}
    \begin{itemize}
        \item The security of RSA encryption hinges on the challenge of factoring the product of two large prime numbers.
        \item Today, known algorithms would require an infeasible amount of time, extending beyond the age of the universe, to factorize sufficiently large primes, ensuring our security.
        \item The practicality of RSA encryption is rooted in the computational simplicity of multiplying numbers, a task easily handled by computers.
        \item For instance, computing $31 \cdot 37 = 1147$ is straightforward, but finding the factors of $1147$ without prior knowledge of its construction is significantly more time-consuming.
    \end{itemize}
\end{frame}

\begin{frame}{RSA Algorithm}
    \framesubtitle{The Setup}
    \begin{enumerate}
        \item The receiver first selects two large prime numbers, $p$ and $q$. Their product, $n = pq$, forms part of the public key.
        \item Next, the receiver computes Euler's totient function: $\phi(n) = \phi(pq) = \phi(p)\phi(q) = (p-1)(q-1)$. Then, they choose a number $e$ coprime to $\phi(n)$. This $e$ constitutes the rest of the public key.
        \item Finally, the receiver calculates the modular inverse $d$ of $e$ modulo $\phi(n)$. This $d$ serves as the private key. Computing $d$ is computationally challenging without knowing $n = pq$, as it requires the knowledge of $\phi(n) = (p-1)(q-1)$. 
    \end{enumerate}
\end{frame}

\begin{frame}{RSA Algorithm}
    \framesubtitle{Transmitting Messages}
    The receiver broadcasts their public key $(n, e)$ and keeps their private key $d$ confidential.
    \begin{enumerate}
        \item The sender converts their message into a number $m$, typically using a system like ASCII encoding. It's important that $m < n$ to avoid losing the message in the encryption process when taking modulo $n$.
        \item Next, the sender computes the encrypted ciphertext: $c \equiv m^e \pmod{n}$. This ciphertext, along with the public key, is the only information accessible to a potential attacker.
        \item Upon receiving the ciphertext, the receiver decrypts it to retrieve the original message: $m \equiv c^{d} \pmod{n}$.
    \end{enumerate}
    Step 3 relies on Euler's theorem, which states that $m^{\phi(n)} \equiv 1 \pmod{n}$, and the choice of $d$ such that $de \equiv 1 \pmod{\phi(n)}$. This ensures the existence of an integer $k$ such that $de = k\phi(n) + 1$. Consequently,
    \[
        m^{de} \equiv m^{k\phi(n) + 1} \equiv m^{k\phi(n)} \cdot m \equiv (m^{\phi(n)})^k \cdot m \equiv 1^k \cdot m \equiv m \pmod{n}.
    \]
\end{frame}

\begin{frame}{Example}
    \framesubtitle{The Setup}
    Let's have the receiver select primes $p = 13$ and $q = 17$. In practice, primes would be much larger to avoid falling victim to brute force attacks.
    \begin{enumerate}
        \item $n = pq = 13 \times 17 = 221$, which is half of the public key.
        \item $\phi(n) = (13 - 1)(17 - 1) = 192$. The receiver chooses $e = 5$.
        \item Then, the receiver calculates $d = 77$, since $de \equiv 1 \pmod{\phi(n)}$.
        \item Finally, the receiver distributes their public key $(221, 5)$.
    \end{enumerate}
\end{frame}

\begin{frame}{Example}
    \framesubtitle{Message Transmission}
    Note that the public key is $(221, 5)$.
    
    Let's say the sender wants to transmit the message "DAVID". We can use an ASCII table to convert each character to its ASCII number. So we have $m = 6869766868$. However, note that $m > n$, which would cause our message to be lost if we try to send it all at once. Therefore, we send our message piece by piece.

    \begin{enumerate}
        \item Starting with the letter "D", we have $m = 68$.
        \item To encrypt the message $m$, the sender calculates $c \equiv m^e \equiv 68^5 \equiv 204 \pmod{221}$. Thus $c = 204$ and it is sent to the receiver. Any third-party attackers will also be able to see this encrypted message.
        \item The receiver, however, will be able to decrypt it with their private key $d = 77$. $c^{d} = 204^{77} \equiv 68 \pmod{221}$, thus getting the message $m = 68$.
        \item The receiver translates this to the letter "D".
    \end{enumerate}
\end{frame}

\begin{frame}{Applications}
    \begin{itemize}
        \item Online Banking
        \item Account logins
        \item Digital Signatures
        \item VPN Encryption
        \item Secure Shell (remote file access and transfer)
        \item Email encryption
        \item Web browser authentification
        \item Token-based logins
    \end{itemize}
\end{frame}


\begin{frame}{Vulnerabilities}
    \begin{itemize}
        \item The strength of RSA is measured by the number of bits in $n$. To prevent brute force attacks, most newly generated keys are $4096$ bits long. A $512$-bit RSA key can be cracked in a few hours, and in 2010, a brute force attack on a $768$-bit key was successful.
        \item Computers can easily compute the greatest common divisor of two numbers using the Euclidean algorithm. Thus, an attacker can use this algorithm on two public keys. If their greatest common divisor is not 1, then the attacker has found a prime number dividing both keys, compromising both keys and allowing the attacker to promptly find the private key.
        \item If $p$ and $q$ are close together such that $m^{e} < n$, the attacker can determine the private key efficiently.
        \item Quantum computers can factor numbers exponentially faster than current algorithms (Shor's algorithm). However, as of today, quantum computers are too small to factor numbers greater than 16-bits.
    \end{itemize}
\end{frame}





    
    
    

\end{document}